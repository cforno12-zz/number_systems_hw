\documentclass[12pt]{article}
\usepackage{amssymb}

\title{MATH 330 -- HW \#25}
\author{Cristobal Forno}
\date{\today}

\begin{document}

\maketitle

\textbf{Proposition 11.7}: The rational number $\frac{m}{n} \in \mathbb{Q}$ is
positive if and only if either $m > 0$ and $n > 0$, or $m < 0$ and $n < 0$.

\textbf{Proof:} Assume that $m > 0$ and $n > 0$, where $m, n \in \mathbb{Z}$.
Proposition 8.40(i) implies that $n$, $\frac{1}{n}$, $m$, and $\frac{1}{m}$ all
have the same sign. Since both $m$ and $n$ are positive, then $\frac{m}{n} = m \cdot
\frac{1}{n} > 0$, by Axiom 8.26(ii). If $m$ and $\frac{1}{n}$ are both negative
then $\frac{m}{n} > 0$ by Prop. 8.32(iii). In both cases, $\frac{m}{n}$ is positive.

Conversely, suppose that $m$ and $n$ have opposite signs. Again, Proposition
8.40(i) implies that $n$ and $\frac{1}{n}$ have the same sign, so $m$ and
$\frac{1}{n}$ have opposite signs. If $m > 0$ and $\frac{1}{n} < 0$, then
$\frac{m}{n} < 0$ and if $m < 0$ and $\frac{1}{n} > 0$, then $\frac{m}{n} < 0$,
both by Prop. 8.32(ii). In both cases, $\frac{m}{n}$ is negative. $\Box$

\end{document}