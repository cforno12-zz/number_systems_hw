\documentclass[12pt]{article}
\usepackage{amssymb}

\title{MATH 330 -- HW \#19}
\author{Cristobal Forno}
\date{11/05/2017}

\begin{document}

\maketitle

\noindent
\textbf{Proposition 9.15}: If $k \in \mathbb{N}$, then $e(k) \in
\mathbb{R}_{>0}$.\\

\noindent
\textbf{Proof}:\\
Assume $k \in \mathbb{N}$. By Axiom 2.1, $k \in \mathbb(Z_{>0})$ as well. Axiom
9.2(i) implies that $e(k) \in \mathbb{R}_{>0}$. Thus, proving our statement if $k \in \mathbb{N}$, then $e(k) \in
\mathbb{R}_{>0}$. $\Box$\\

\noindent
\textbf{Proposition 9.18}: The function $e$ preserves multiplication: for all
$m, k \in \mathbb{Z}$,
\begin{center}
  $e(m \cdot k) = e(m) \cdot e(k)$, 
\end{center}
where $\cdot$ on the left-hand side refers to multiplication in $\mathbb{Z}$,
whereas $\cdot$ on the right-hand side refers to multiplication in $\mathbb{R}$.\\
\noindent
\textbf{Proof}:\\
Let $m \in \mathbb{Z}$. We first show that for all $n \in \mathbb(Z_{>0})$ that
$e(m \cdot n) = e(m) \cdot e(n)$ by induction on $n$. First, when $n=0$,
\begin{center}
  $e(m \cdot n) = e(m \cdot 0) = e(0) = 0 = e(m) \cdot 0 = e(m) \cdot e(0) =
  e(m) \cdot e(n)$.
\end{center}
Now suppose for some $n \geq 0$ that $e(m \cdot n) = e(m) \cdot e(n)$. By induction hypothesis,
\begin{center}
  $e(m \cdot (n+1)) = e(m \cdot n + m)$ \\ $= e(m \cdot n) + e(m)$ \\ $= e(m)
  \cdot (e(n) + 1)$ \\ $= e(m) \cdot e(n + 1)$.
\end{center}
\noindent
This completes our proof by induction.\\
It remains to show that for all $n < 0$, $e(m \cdot n) = e(m) \cdot e(m)$.
Suppose $n < 0$. Then $-n > 0$. So,
\begin{center}
  $e(m \cdot n) = e(-(m(-n))) = -e(m(-n)) = -e(m) \cdot e(-n) = e(m) \cdot e(n)$.
\end{center}
$\Box$
\end{document}

