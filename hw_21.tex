\documentclass[12pt]{article}
\usepackage{amssymb}

\title{MATH 330 -- HW \#21}
\author{Cristobal Forno}
\date{\today}

\begin{document}

\maketitle

\noindent
Let $A$ be a subset of $\mathbb{R}$. We define $-A = \{-a : a \in A\}$.\\
\noindent
\textbf{Question} Let $A$ be bounded subset of $\mathbb{R}$. Prove that \\
\textbf{(1)} inf($-A$) = $-$sup($A$)\\
\indent
\textbf{Proof}: We want to prove inf($-A$) $\leq$ $-$sup($A$) \textbf{(a)} and inf($-A$)
$\geq$ $-$sup($A$) \textbf{(b)} then it implies that inf($-A$) $=$ $-$sup($A$). \\
\indent \indent
\textbf{(a)}We know the inf($-A$) $\leq$ for all $-a \in -A$. Therefore,
$-$inf($-A$) $\geq$ $a$. In other words, $-$inf($-A$) is an upper bound for $A$.
Hence, sup($A$) $\leq$ $-$inf($-A$), which is equivalent to, inf($-A$) $\leq$
$-$sup($A$).\\
\indent \indent
\textbf{(b)} Since $A$ is bounded above, then $a \leq$ sup($A$). We can compute
that, $-a \geq -$sup($A$). Therefore, $-$sup($A$) is a lower bound for $-A$ which implies that
inf($-A$) $\geq$ $-$sup($A$).\\
\indent
Finally, since inf($-A$) $\leq$ $-$sup($A$)  and inf($-A$)
$\geq$ $-$sup($A$) implies that inf($-A$) $=$ $-$sup($A$). $\Box$\\

\textbf{(2)} sup($-A$) = $-$inf($A$) \\
\indent
\textbf{Proof}: We want to prove sup($-A$) $\leq$ $-$inf($A$) \textbf{(a)} and sup($-A$)
$\geq$ $-$inf($A$) \textbf{(b)} then it implies that inf($-A$) $=$ $-$sup($A$). \\
\indent \indent
\textbf{(a)} We know the sup($-A$) $\geq$ for all $-a \in -A$. Therefore,
$-$sup($-A$) $\leq$ $a$. In other words, $-$sup($-A$) is a lower bound for $A$.
Hence, $-$sup($-A$) $\leq$ inf($A$), which is equivalent to, sup($-A$) $\leq$
$-$inf($A$).\\
\indent \indent
\textbf{(b)} Since $A$ is bounded below, then inf($A$) $\leq$ for all $a \in A$.
We can compute that, $-$inf($A$) $\geq$ $-a$. Therefore, $-$inf($A$) is an
upperbound for $-A$ which implies that sup($-A$) $\geq$ $-$inf($A$).\\
\indent
Finally, since \textbf{(a)} and \textbf{(b)} hold, then both statements imply that sup($-A$) = $-$inf($A$). $\Box$

\end{document}

