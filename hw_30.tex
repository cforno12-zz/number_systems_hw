\documentclass[12pt]{article}
\usepackage{amssymb}
\usepackage{amsmath}

\title{MATH 330 -- HW \#30}
\author{Cristobal Forno}
\date{\today}

\begin{document}

\maketitle

\textbf{Question:} Let $A = \mathbb{R} \diagdown \{-1\}$, that is, $A$ is the set of all
integers excluding $-1$. We define a binary operation $\circ$ on $A$ as follows:
\begin{center}
  $a \circ b = ab +a +b$, for all $a,b \in A$
\end{center}
Verify the following:
\newline
(1) $a \circ b \in \mathbb{A}$ for all $a,b \in \mathbb{A}$.

\noindent
\textbf{Proof:}
$A$ is closed if $a, b \in A$ implies: $a \circ b \in A$. Suppse $a \circ b = ab +
a + b = -1$. This means $ab + a + b +1 = (a+1) (b+1) = 0$. So, either $a = -1$ or
$b = -1$. $\Box$

\noindent
(2) for $a, b, c \in A, (a \circ b) \circ c = a \circ (b \circ c)$

\noindent
\textbf{Proof:}
The operation $\circ$ is associative if and only if:
\begin{center}
  $(a \circ b) \circ c = a \circ (b \circ c)$ for all $a, b, c \in A$
\end{center}
Let's computer both sides of this equation:
\begin{center}
  $(a \circ b) \circ c = (ab + a + b) \circ c = (ab+a+b)c + (ab+a+b) + c =
  abc+ac+bc+ab+a+b+c$\\
  $a \circ (b \circ c) = a \circ (bc + b +c) = a(bc+b+c)+a+(bc+b+c) =
  abc+ac+bc+ab+a+b+c$
\end{center}
Since both equations are the same, the operation $\circ$ is associative.$\Box$
  
\noindent
(3) $a \circ 0 = a = 0 \circ a$ for all $a \in A$

\noindent
\textbf{Proof:}
We want to the operation $\circ$ has an identity element $0$. We must show that $a \circ 0 = a = 0 \circ a$.\\
The left side of the equation $a \circ 0 = a(0)+a +0 = a$.\\
The right right of the equation $0 \circ a = (0)a + 0 + a = a$.\\
Since both equations are the same, $a \circ 0 = a = 0 \circ a$. $\Box$ \\

\noindent
(4) for each $a \in A$, there exits $b \in A$ such that $a \circ b = 0 = b \circ
a$

\noindent
\textbf{Proof:}
For a given $a \in A$, we need to find $b \in A$ such that
\begin{center}
  $a \circ b = 0 = b \circ a$
\end{center}

(1) We will first solve for $b$ for the equation on the left.
\begin{align}
  a \circ b = ab + a + b &= 0\\
  = b(a-1) &= -a
\end{align}
As $a \in A$, $a \neq -1$, so $a+1 \neq 0$. So $b = -\frac{a}{a+1}$.
By a way of contradiction, we will assume $b = -1$.
\begin{align}
  -\frac{a}{a+1} &= -1 \\
  -a &= -(a+1) \\
  -a &= -a-1 \\
  0 &\neq 1
\end{align}
So $b = -\frac{a}{a+1} \neq 1$ or $b \in A$.
(2) We will now solve for $b$ for the right side of the equation.
\begin{align}
  b \circ a &= -\frac{a}{a+1}a - \frac{a}{a+1}+a \\
            &= -\frac{a^2-a}{a+1}+a \\
            &= \frac{(-a)(a+1)}{1 \times (a+1)} +a \\
            &= -a + a = 0
\end{align}
Thus, $b$ is the inverse of $a \in A$. $\Box$


\noindent
(5) $a \circ b = b \circ a$ for all $a, b \in A$.

\noindent
\textbf{Proof:}
For all $a,b \in A$,
\begin{center}
  $a \circ b = b \circ a$
\end{center}
So we have,
\begin{align}
  (a \circ b) - (b \circ a) &= 0 \\
  (ab + a + b) - (ba + b + a) &= 0 \\
  ab + a + b - ba - b -a &= 0 \\
  0 &= 0
\end{align}
$\Box$

\noindent
Thus, we conclude that $(A, \circ)$ is an abelian group with $0$ as an identity element.
\end{document}