\documentclass[12pt]{article}
\usepackage{amssymb}

\title{MATH 330 -- HW \#22}
\author{Cristobal Forno}
\date{\today}

\begin{document}

\maketitle

\textbf{Proposition 10.21}: Let $L = \lim_{k\to\infty} x_k.$\\
$(i)$ If $(x_k)_{k=1}^\infty$ is increasing, then $x_k \leq L$ for all $k \geq 0$.\\
\noindent
\textbf{Proof}:\\
\indent
Assume $(x_k)_{k=1}^\infty$ is increasing and $L = \lim_{k\to\infty} x_k$. We
will prove by contradiction. Assume there exists $m \in \mathbb{N}$ such that $x_m
> L$. Let $\varepsilon = x_m - L$. Then for any $M \in \mathbb{N}$, let k =
$\max\{M, m\}$. So, in particular, $k \geq M$. Since the sequence is increasing,
we have $x_k \leq x_m > L$. But then $|x_k - L| = x_k - L \geq x_m - L =
\varepsilon$. Thus, the sequence $(x_k)_{k=1}^\infty$ does not converge to $L$,
then $L \neq \lim_{k\to\infty} x_k$ (implied by Proposition 10.16), 
which is a contradiction. $\Box$ \\

\noindent
$(ii)$ If $(x_k)_{k=1}^\infty$ is decreasing, then $x_k \geq L$ for all $k \geq 0$.\\
\noindent
\textbf{Proof}:\\
\indent
Assume $(x_k)_{k=1}^\infty$ is decreasing and $L = \lim_{k\to\infty}x_k$. We
will prove by contradiction. Assume there exists $m \in \mathbb{N}$ such that
$x_m < L$. Let $\varepsilon = L + x_m$. Then for any $M \in \mathbb{N}$, let k =
$\max\{M, m\}$. So, in particular, $k \leq M$. Since the sequence is decreasing,
we have $x_k \geq x_m < L$. But then $|L+x_k| =  L+x_k \leq L+x_m=\varepsilon$.
Thus, the sequence $(x_k)_{k=1}^\infty$ does not converge to $L$, then $L \neq
\lim_{k\to\infty} x_k$ (implied by Proposition 10.16), which is a contradiction. $\Box$

\end{document}