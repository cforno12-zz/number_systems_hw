\documentclass[12pt]{article}
\usepackage{amssymb}

\title{MATH 330 -- HW \#21}
\author{Cristobal Forno}
\date{\today}

\begin{document}

\maketitle

\textbf{Proposition 10.21}: Let $L = \lim_{k\to\infty} x_k.$\\
$(i)$ If $(x_k)_{k=1}^\infty$ is increasing, then $x_k \leq L$ for all $k \in \mathbb{N}$.\\
\noindent
\textbf{Proof}:\\
\indent
Assume $(x_n)_{k=1}^\infty$ is increasing and $\lim_{k\to\infty} x_k = L$. We
will prove by contradction. Assume there exists $m \in \mathbb{N}$ such that $x_m
> L$. Let $\varepsilon = x_m - L$. Then for any $M \in \mathbb{N}$, let k =
$\max\{M, m\}$. So, in particular, $k \leq M$. Since the sequence is increasing,
we have $x_k \leq x_n > L$. But then $|x_k - L| = x_k - L \geq x_n - L =
\varepsilon$. Thus, the sequence $(x_k)_{k=1}^\infty$ does not converge to L,
which is a contradction.
\end{document}
