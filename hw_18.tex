\documentclass[12pt]{article}
\usepackage{amssymb}

\title{MATH 330 -- HW \#18}
\author{Cristobal Forno}
\date{10/29/2017}

\begin{document}

\maketitle

\noindent
\textbf{Proposition 9.11}: If a function is bijective then its inverse is unique.

\noindent
\textbf{Proof}:\\
Let $f$: be a bijective function from  $A \to B$.
Suppose $g_1$ and $g_2$ are both inverses to $f$. Then,
\begin{center}
$ g_1 = g_1 \circ i_B = g_2 \circ (f \circ g_2) = (g_1 \circ f) \circ
g_2 = i_A \circ g_2 = g_2 $,
\end{center}
\noindent
proving that there is only one unique inverse for a function. $\Box$ \\

\noindent
\textbf{Proposition 9.12}: Let $A$ and $B$ be sets. There exists an injection
from $A \to B$ if and only if there exists a surjection from $B \to A$.

\noindent
\textbf{Proof}:\\
We want to prove if there exists an injection from $A \to B$, then there exits a
surjection from $B \to A$ and if there exits a surjection from $B \to A$,
then there exits an injection from $A \to B$. Suppose $f:A \to B$ is an
injection. Then by Proposition 9.10 (i), $f$ has a left inverse $g: B \to A$. So,
\begin{center}
$g \circ f = id_A$
\end{center}
This implies that $g$ has a right inverse, and thus $g$ is surjective by
Proposition 9.10 (ii). Similary, if $g: B \to A$ is surjective, then $g$ has a
right inverse $f: A \to B$. Thus, $f$ has a left inverse, $f$ is injective. $\Box$

\end{document}

